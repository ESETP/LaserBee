\documentclass[a4paper]{article}
\usepackage[cm]{fullpage}
%\usepackage{txfonts}
\usepackage{listings}
\usepackage{hyperref}

\hypersetup{
	colorlinks=true,
}

\title{Hiasynth Microcontroller Reference} % going to have to check the spelling on this one big man
\author{Team 1}

\begin{document}

\maketitle

\section{Microcontroller details}
% list the name and precice package of the uc, where you can find the source code, and what each pin is assigned to.

\section{Flashing the efm8 }
\subsection{Building the project}
During development, the project was built using the Keil c51 toolchain through SiLab's Simplicity Studio. The full compliment of files to enable this will be provided. 

The project can be imported in to simplicity studio using the import wizard under File, and built using the hammer icon. If successful, this should produce .hex files under \lstinline|project/Keil 8051 [version] Debug/|. You will need this file later, so don't lose it.

\subsection{Converting}
As far as my testing goes, this part of the process, and flashing, only works on Windows. See the Linux issues section for more.

The next parts of the process require some more tools that Silabs provides. These can be downloaded \href{https://www.silabs.com/documents/public/example-code/AN945SW.zip}{here.}

You need to convert the raw .hex file to a .efm8 before flashing, using the tool hex2boot.exe, located under \lstinline|/AN945SW/Tools/Windows/|. Run this using powershell or cmd with a command something like this \lstinline|.\hex2boot.exe input_file.hex -o Filename.efm8 |. The full details and documentation of this utility are available in section 6 \href{https://www.silabs.com/documents/public/application-notes/an945-efm8-factory-bootloader-user-guide.pdf}{here.}

\subsection{Flashing}
\subsubsection{Ingredients:}
\begin{itemize}
	\item Windows computer
	\item Usb to serial programmer, also known as an FTDI thingy,
	\item Wires
	\item .efm8 file
	\item Hiasynth
	\item efm8load.exe tool
\end{itemize}

\subsubsection{Method:}
\begin{enumerate}
	\item Connect the FTDI programmer to the laser bee. Currently it's % insert
	\item Plug the programmer into the computer, and find out what serial port it is attached to. 
	\item Set the Hiasynth to bootloader mode. This can be done by grounding the C2D pin (3.7, or the second pin from the left on the header) as the device is reset. 
	\item Run efm8load.exe with a command something like \lstinline|./efm8load.exe -p PORT Filename.efm8|
	\item Reset the hiasynth and enjoy
\end{enumerate}

\section{Linux issues}

I attempted to get this working on linux, but ran into 2 big issues: hex2boot, and actually flashing it. Hex2boot is only distributed as an exe, with no source files. One person on github reports success running it under wine, but I was not able to get it going. In addition I tried a third party hex2boot converter, and this also did not work. As for flashing it, SiLabs distributes the python source files for efm8flash, but I was not able to get these working either. It requires some modification (detailed \href{https://community.silabs.com/s/article/how-to-use-efm8-uart-bootloader-on-linux?language=en_US}{here.}), and some difficult to source libraries, and despite following the instructions, it did not work for me.



\end{document}